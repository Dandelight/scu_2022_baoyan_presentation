% LTeX: enabled=false
\documentclass[lualatex,aspectratio=169,serif]{beamer}

% ~~~~~~~~~~~~~~~~~~~~~
% ~ Required packages ~
% ~~~~~~~~~~~~~~~~~~~~~
\usepackage[noindent,UTF8]{ctexcap}
% Theme
\usepackage{theme/beamerthemematerial}
% Roboto font
\usepackage{fontspec}
\usepackage[sfdefault]{roboto}
% Language specific typesetting rules
\usepackage[english]{babel}
% Required for source/credits below pictures
\usepackage[absolute,overlay]{textpos}
% For colorful layouts
\usepackage{xcolor, etoolbox, xstring}
\usepackage{tikz,tcolorbox}
\usetikzlibrary{shapes,calc}
\tcbuselibrary{skins}

\setbeamertemplate{theorems}[numbered]
\definecolor{myblue}{rgb}{0.15,0.15,0.53}

% 1- Block title (background and text)
\setbeamercolor{block title example}{fg=white, bg=teal}
% 2- Block body (background and text)
\setbeamercolor{block body example}{ bg=teal!25}
% Change alert block colors
% 1- Block title (background and text)
\setbeamercolor{block title alerted}{fg=white, bg=orange}
% 2- Block body (background and text)
\setbeamercolor{block body alerted}{ bg=orange!25}
% Change standard block colors
% 1- Block title (background and text)
\setbeamercolor{block title}{bg=cyan, fg=white}
% 2- Block body (background)
\setbeamercolor{block body}{bg=cyan!10}

\setCJKsansfont[ItalicFont=KaiTi]{Microsoft Yahei}
\renewcommand{\CJKfamilydefault}{\CJKsfdefault}

\makeatother

% ~~~~~~~~~~~~~~~~~~~~~~~
% ~ Additional packages ~
% ~~~~~~~~~~~~~~~~~~~~~~~
\usepackage{csquotes}
\usepackage{listings}


% ~~~~~~~~~~~~~~~~~~~~~
% ~ Required packages ~
% ~~~~~~~~~~~~~~~~~~~~~
\usepackage{hyperref}

% ~~~~~~~~~~~~~~~~~~~~~~~
% ~ Theme configuration ~
% ~~~~~~~~~~~~~~~~~~~~~~~
% \beamertemplatenavigationsymbolsempty
\setbeamertemplate{navigation symbols}{}
\addtobeamertemplate{navigation symbols}{%
    \usebeamerfont{footline}%
    \usebeamercolor[fg]{footline}%
	% \strut
	\inserttitle
	\hspace{1em}
	\insertsubtitle
	\hspace{1em}
	% \insertshortinstitute
	\insertinstitute
	\hspace{1em}
	\insertauthor
	\hspace{1em}
    \insertframenumber/\inserttotalframenumber
}{}
% LTex: enabled=true

\usepackage{pgf-pie} % the package used to implement the pie charts

\definecolor{Accent}{HTML}{1976D2}
\renewcommand{\emph}[1]{{\color{Accent}{\textbf{#1}}}}

\title{学习与保研经验分享}
\subtitle{CCF川大分会``对话菁英''经验交流活动}
\date{2022年11月13日}
\author[R. Guo]{郭睿明}
\institute[SCU]{四川大学 计算机学院}

\begin{document}

% ~~~~~~~~~~~~~~~~~~~~~~~
% ~ Presentation slides ~
% ~~~~~~~~~~~~~~~~~~~~~~~

\begin{frame}
	\titlepage
\end{frame}

{
% Link colors are rendered with `AccentColor'
% Change the AccentColor temporarily to match the background
\colorlet{accent-color}{secondary-color-text}
\begin{frame}{目录}
	\tableofcontents
\end{frame}
}

\begin{frame}
	\frametitle{免责声明}
	\framesubtitle{Disclaimer}
	\centering
	本宣讲中的内容\emph{仅为2022年}四川大学计算机学院参与保研的学生的经验总结,不代表理解全面准确,也不代表之后年份的情况会完全相同,仅供参考。一切事宜以学院、学校解释为准。
\end{frame}

\begin{frame}
	\frametitle{为什么要读研}

	\begin{itemize}
		\item 对学术的热爱
		\item 提升自身实力
		\item 社会对学位的认可与要求
	\end{itemize}

\end{frame}

\section{保研流程}
\topicFramePrimary{保研流程}
\begin{frame}{保研是什么}
	% 因为现场考虑到大一新生,可能入校前后已经或多或少听说过了,这里简单带过一下
	保研全称\emph{推荐优秀应届本科毕业生免试攻读硕士学位研究生},
	本科学生可以通过成绩评定、综合排名等形式,获得推免资格,不经过考研(全国硕士研究生统一招生考试)而直接被高校录取为研究生。

	% 前一位考研的学长可以成功上岸非常优秀,但是保研还是一个更推荐的选项,为什么呢,
	保研有以下优势:
	\begin{itemize}
		\item 更广的选择面:从4-5月一直到9月总有机会获得接受资格;可以报名多所院校的选拔
		\item 节约时间:保研结果在9月底尘埃落定,大四有充足的自由时间
		\item 提前联系导师:提前``占坑'',获得更优质的指导
		\item 直博:只有保研学生可以申请直博,免去了硕士的毕业事务
	\end{itemize}
\end{frame}

\begin{frame}
	\frametitle{保研流程}
	\framesubtitle{以及各方关系}
	保研涉及三方:

	\begin{itemize}
		\item \emph{学生};
		\item \emph{推荐院校},即学生本科就读院校;
		\item \emph{接收院校},即学生意向研究生院校。
	\end{itemize}
	后两者可能相同。

	\begin{itemize}
		\item 表面的流程是:本科院校推荐 $\rightarrow$ 统一报名 $\rightarrow$ 接收院校考核
		\item 实际的流程是:接收院校夏令营(5-8月) $\rightarrow$ 接收院校``预推免''(8-9月)$\rightarrow$ 推荐院校进行``综测'',学生获得保研名额(9月开学后)$\rightarrow$ 学生在系统中填报,接收院校在系统中接收(9月28日)
	\end{itemize}
\end{frame}

\begin{frame}
	\frametitle{问题分解}
	所以保研问题就是三者之间的关系问题:
	\begin{itemize}
		\item 学生-推荐院校:在综测排名中拔得头筹
		\item 学生-接收院校:在夏令营/预推免中脱颖而出
		\item 推荐院校-接受院校:档案交接等
	\end{itemize}
	% 其中推荐院校和接受院校的关系一部分还是需要我们亲力亲为的,比如成绩单、推荐信等申请材料的递交
\end{frame}

\section{推荐院校}
\topicFramePrimary{推荐院校}

\begin{frame}
	\frametitle{计算机学院推荐政策(2022年)}
	\begin{columns}
		\begin{column}{0.4\linewidth}
			\begin{itemize}
				\item \emph{课程成绩}
				      \begin{equation}
					      0.6\times \frac{\sum_i \mathit{score}_i \times \mathit{grade\_point}_i}{\sum_i 100\times \mathit{grade\_point}_i}
				      \end{equation}
				\item \emph{科研潜质}:六大维度
				\item \emph{社会活动}:优秀学生、优秀志愿者、优秀学生干部等
				      % 注意为啥是优秀?
			\end{itemize}
			按照综合测评成绩排名,每专业取若干名同学授予推免资格。全学院有一定数量候补同学。
		\end{column}

		\begin{column}{0.6\linewidth}
			\begin{tikzpicture} % Tikz environment
				\pie[rotate=30]
				% it is essential to recheck that the sum of all the components mentioned should be 100%. Otherwise, Latex will leave the left percentage, blank.
				{60/课程成绩, 35/科研潜质, 5/社会活动}
			\end{tikzpicture}
		\end{column}
	\end{columns}
\end{frame}

\begin{frame}
	\frametitle{课程成绩}
	计算机是工科,学习课本知识和编码实践同样重要。

	推荐网课来源:
	\begin{enumerate}
		\item \url{https://csdiy.wiki} 一个国内外名校课程目录,内容涵盖计算机科学方方面面,只有想不到没有学不到
		\item 中国大学 MOOC
		\item 哔哩哔哩
		\item Coursera
		\item 一些优秀的开源项目,比如 rCore \url{https://rcore-os.cn/}
	\end{enumerate}

	还有一些优秀的参考资料。中文参考资料往往言简意赅适合备考,英文资料翔实丰富适合自学。

	根据培养计划,计算机学院各专业从大一到大三都有必修课,大一大二学分较多,但大三也有不少。准备保研中,注意平衡课程成绩和其他加分项。一般来说,科研成绩优异的同学,课程成绩也不会差。
\end{frame}

\begin{frame}
	\frametitle{科研创新潜质与专业能力倾向}
	\begin{columns}
		\begin{column}{0.5\linewidth}
			\begin{center}
				\begin{tikzpicture}
					\newdimen\R
					\R=2.2cm
					\draw (0:\R) \foreach \x in {60,120,...,360} {  -- (\x:\R) };
					\foreach \x/\l/\p in
					{ 60/{\emph{专业竞赛}}/above,
					120/{\emph{学术论文}}/above,
					180/{\emph{科研活动}}/left,
					240/{\emph{知识产权}}/below,
					300/{\emph{专业认证}}/below,
					360/{\emph{科研成果}}/right
					}
					\node[inner sep=1pt,circle,draw,fill,label={\p:\l}] at (\x:\R) {};
				\end{tikzpicture}
			\end{center}
		\end{column}
		\begin{column}{0.5\linewidth}
			\begin{itemize}
				\item 各维度评分:令第 $i$ 位学生六个维度的每个维度分别为 $s_{i1}, s_{i2}, \ldots, s_{i6}$,$s_{ij}$ 为第 $j$ 各项目可选成果中 $i$ 生加分最高一项。
				\item 综合赋分:令 $S_i = \sum_{j=1}^6 s_{ij}$ 为第 $i$ 位参与推免的同学的六个维度评分之和,则该同学在科研潜质板块获得的得分为
				      \begin{equation}
					      C_i = \frac{S_i}{\max\{S_1, S_2, \ldots, S_n\}}
				      \end{equation}
			\end{itemize}
		\end{column}
	\end{columns}

	\footnote{计算机学院科研创新潜质和专业能力倾向成绩计分细则}
\end{frame}

\begin{frame}
	\begin{enumerate}
		\item 专业竞赛:按照参赛范围分为国际级、国家级、省级、校级;根据赛事权威性分为 I、II、III、IV 四级;
		\item 学术论文:{CCF 期刊会议分级}\footnote{\url{https://www.ccf.org.cn/Academic\_Evaluation/By\_category/}}、川大期刊分级,只认可该生为第一作者、四川大学为论文第一单位的文章;
		\item 科研活动:按国家级、省级、校级、院级划分,成员的加分是负责人的一半。大创也算科研活动。事实上,很多学生通过大创加上了不少分;
		\item 知识产权:目前只认可国家发明专利和国际专利。第一署名人90分,其它署名人10分。四川大学必须是第一署名单位和权利人;
		\item 专业认证:虽然其他国际认证、计算机技术与软件专业技术资格(水平)考试都可以,但 CCF CSP 必须推荐大家准备;
		\item 科研成果:省部级以上科研成果奖。% 我还没看见有哪个本科生得过这个奖。
	\end{enumerate}
\end{frame}

\begin{frame}
	\frametitle{可选路径}
	\begin{itemize}
		\item 科研:大创(科学探索类) $\rightarrow$ 论文 + 专利
		\item 工程:大创(工程技术类) $\rightarrow$ 专利 + EI 论文
		\item 竞赛:竞赛 $\rightarrow$ 参与到老师的课题组 $\rightarrow$ 大创 (回到 1、2)
		\item CSP:竞赛 $\rightarrow$ CCF CSP
	\end{itemize}

\end{frame}

\begin{frame}
	\frametitle{社会实践活动}

	\begin{enumerate}
		\item 优秀学生、优秀学生干部
		\item 社会活动
		      \begin{enumerate}
			      \item 社会实践优秀个人
			      \item 优秀团员、团干部
			      \item “百佳”文明班级/寝室/个人
			      \item 学生组织服务核心工作
		      \end{enumerate}
		\item 课外学生活动、非专业类竞赛获奖
		      \begin{enumerate}
			      \item 创新创业竞赛
			      \item 实践操作技能竞赛
			      \item 演讲赛、辩论赛
			      \item 文体竞赛
		      \end{enumerate}
		\item 校内外公益活动(获校级及以上奖励)
	\end{enumerate}
\end{frame}


\section{接收院校}
\topicFramePrimary{接收院校}

\begin{frame}
	\frametitle{接收情况概述}
	\emph{夏令营}和\emph{预推免}已经成为推免招生的最主要途径。

	二者由接收院校自行组织\footnote{一般是学校提供平台,学院自己组织内容、组织筛选}。

	夏令营期间,除校领导介绍学校、导师介绍课题组之外,还会对参营学院进行\emph{笔试、面试等考核,颁发《优秀营员》证书},即获得该学校该专业的推免资格\footnote{不同学校所发证书的名目有所不同,本文中将获得接收推免资格的学生统称为“优秀营员”,简称“优营”。}。

	自疫情开始,夏令营几乎全部为线上进行,但随着防疫政策的动态调整,近几年可能有一部分会转为线下。

	\emph{流程}:资料准备 $\rightarrow$ 信息获取+准备考核 $\rightarrow$ 报名 $\rightarrow$ 参与考核 $\rightarrow$ 系统填报
\end{frame}

\begin{frame}
	\frametitle{一些通常说法}

	\begin{itemize}
		\item pro:Professor,即导师
		\item com,Committee,即招生委员会
		\item 强/弱com:指招生委员会与导师在招生上的相对权利
		\item offer,即录取通知=优营
		\item wl,Wait list,候补名单
		\item rk,Rank,即排名
		\item bar,门槛
		\item oq,uq,Over Qualified, Under Qualified
		\item 鸽,放鸽子的简称
	\end{itemize}

\end{frame}

\begin{frame}{需要准备的材料}
	\begin{enumerate}
		\item 基本信息
		      \begin{enumerate}
			      \item 成绩排名证明(学院统一出具)
			      \item 成绩单(教务系统导出)、四六级
			      \item 个人信息(因为要填很多遍,所以还是整理出来复制粘贴)
			      \item \emph{申请理由}(要认真写)
			      \item 联系方式(\emph{务必保证时刻畅通})
		      \end{enumerate}
		\item 科研潜质证明
		      \begin{enumerate}
			      \item 奖项证明
			      \item 专利授权/公开证明
			      \item 论文接收证明及全文
		      \end{enumerate}
		\item 其他证明材料
		      \begin{enumerate}
			      \item (可能,视接收院校要求)体检表、政审表
			      \item (部分学校需要一两封)副高级以上教授推荐信
			      \item (部分直博需要)攻读博士期间科研计划
		      \end{enumerate}
	\end{enumerate}
\end{frame}

\begin{frame}
	\frametitle{提前了解老师}

	如果已经有希望从事的研究方向,可以提前找对应方向的老师进行沟通:

	\begin{enumerate}
		\item 在一些“强com”学校,通过委员会选拔之后,再与老师联系即可;
		\item 一些“强pro”学校,提前和老师联系,可能会对选拔有帮助。
	\end{enumerate}

	在本科期间,做的大创、研究中,经常会关注某学者的研究,以这样的学者为目标,对自己的发展也是有好处的。

	如果希望加入优秀导师的课题组,还是应该在前几年\emph{在相关研究领域有所成果},至少是有一定理解。

\end{frame}

\begin{frame}
	\frametitle{信息获取}

	主要途径有
	\begin{itemize}
		\item 学院官网:最佳信息渠道。消息权威,但数量众多,可以将自己感兴趣的几个学院设置为浏览器主页;
		\item 微信公众号:一些公众号会推送近期保研资讯汇总,也有专门针对计算机学生的公众号;
		\item GitHub:一些仓库会更新夏令营、预推免信息。既然是GitHub那必然是专门针对计软网相关学生的;
		\item 同学分享、往届学长等。
	\end{itemize}

	\emph{DEADLINE 很重要!!!不要因为错过DDL而与梦校失之交臂!!!}

\end{frame}

\begin{frame}
	\frametitle{准备考核}

	% 现在本科通过不懈努力,获得成果的学生很多,简单说就是很“卷”。所以也相应地测试难度也提高了,笔试会考比较难的算法题,面试也会问一些专业性很强的问题。

	一般情况下,学院组织统一考核,同期导师/课题组也会单独进行考核。

	笔试/面试前,找好一个安静的角落,确定不会被打扰(在校生,宿舍最好\footnote{也体现了与舍友搞好关系的重要性})

	\begin{center}
		\begin{columns}
			\begin{column}{0.4\linewidth}
				笔试:
				\begin{enumerate}
					\item 刷题练习:力扣、牛客、PTA、CodeForces;
					\item Online Judge
					\item 北航可以用 CSP C/C++ 申请免试。
				\end{enumerate}
			\end{column}

			\begin{column}{0.4\linewidth}
				面试:
				\begin{enumerate}
					\item 做一个简单的 PPT
					\item 针对项目提问
					\item 专业基础知识
					\item 英语能力测试
					\item 未来科研计划
				\end{enumerate}
			\end{column}
		\end{columns}
	\end{center}
	\emph{夏令营/预推免前后一段时间,务必保证预留的联系方式24小时畅通!!!随时准备接收考核相关通知!!!}
	% 手机时刻充电,充电宝随时待命
\end{frame}

\begin{frame}
	\frametitle{填报}

	经过夏令营和预推免,学校公布推免名单后,学生应该已经拿到了心仪的 offer。

	本来系统的意图是方便学校考核,结果学校自己考核,系统反而成了形式。

	几乎所有工作都是在9月28日一天之内完成的。

	系统开放填报前一周,获得推免资格的学生可以进入系统填写个人信息。

	系统开放后,可以报三个意向 $\rightarrow$ 招生单位通知复试 $\rightarrow$ 学生接受复试通知 $\rightarrow$ 学校发放预录取通知\footnote{注意这里根本没有进行复试} $\rightarrow$ 学生接受预录取通知,学生不能主动修改 $\rightarrow$ 提交政审表、体检表等后续材料

\end{frame}

\begin{frame}{科研与工程}{多说两句}

研究更多是在探索未知

工程更多是使用现有技术制造

研究具有探索性和不确定性,过程比较枯燥,需要学生有一定灵感
\end{frame}

\begin{frame}{学硕、专硕与直博}{多说几句}
	\begin{enumerate}
		\item 学术型研究生(学硕)以培养科研人员为主,学习偏向于理论和学术研究;
		\item 专业型研究生(专硕)与学硕处于同一层次,主要是培养特定行业或者特定岗位需求的专业技术型人才。
		\item 博士具备产出原创成果的能力。
	\end{enumerate}
\end{frame}

\section{The End}
\begin{frame}
	\frametitle{The End}
	\centering{Thanks for your attention!}

	\centering{Source for the talk: \url{https://github.com/Dandelight/scu_2022_baoyan_talk}}
\end{frame}

\if0

\begin{frame}[fragile]
	\sidebysideright{
		Configure colors in \texttt{src/config.lua}. Hex color codes are used.

		Colors are exported in the \texttt{src/beamerthemematerial.sty} file.
	}{
		{\Large Mandatory colors}

		\begin{itemize}
			\item[\textcolor{primary-color-text}{\textbullet}] \color{primary-color-text}{Primary}
			\item[\textcolor{secondary-color-text}{\textbullet}] \color{secondary-color-text}{Secondary}
			\item[\textcolor{accent-color}{\textbullet}] \color{accent-color}{Accent}
			\item[\textcolor{bgcolor-01}{\textbullet}] \color{bgcolor-01}{Background-01}
			\item[\textcolor{bgcolor-02}{\textbullet}] \color{bgcolor-02}{Background-02}
			\item[\textcolor{bgcolor-03}{\textbullet}] \color{bgcolor-03}{Background-03}
			\item[\textcolor{bgcolor-04}{\textbullet}] \color{bgcolor-04}{Background-04}
		\end{itemize}
	}
	{halign=center, valign=center}
	{halign=center, valign=center}
\end{frame}

\topicFramePrimary{Features}
\begin{frame}
	\frametitle{Laplacian Eigenmaps}

	\section{Introduction}

	Consider the manifold unfolding problem again: Given a set of points in a high dimensional Euclidean space but along a manifold, $\mathbf{x}$
\end{frame}

\begin{frame}
	\sidebysideleft[0.7]{
		Adjust ratio of the layout with the optional parameter.
		Defaults to half of the page.
	}{
		Right Side
	}
	{halign=center, valign=center}
	{halign=center, valign=center}
\end{frame}

\begin{frame}
	\sidebysideright[0.3]{
		Left Side
	}{
		Right Side\\
		A very important statement
		\begin{equation*}
			a^2 + b^2 = c^2
		\end{equation*}
	}
	{halign=center, valign=center}
	{halign=center, valign=center}
\end{frame}


\section{Usage}
\topicFramePrimary[Explaining how things work]{Usage}
\topicFrameSecondary{Usage}
\subsection{Slides with code}
\begin{frame}[fragile=singleslide]{Code}
	\begin{lstlisting}
def someFunction(x):
	return x*2

someFunction(2)
\end{lstlisting}
\end{frame}

\begin{frame}
	\frametitle{拉普拉斯特征映射}

	\begin{equation}
		\min _{\mathbf{f} \in \mathbb{R}^n:\|\mathbf{f}\|=1} \frac{1}{2} \sum_{i, j} w_{i j}\left(f_i-f_j\right)^2, \quad \text { or } \min _{\mathbf{f} \neq \mathbf{0} \in \mathbb{R}^n} \frac{\frac{1}{2} \sum_{i, j} w_{i j}\left(f_i-f_j\right)^2}{\sum_i f_i^2}
	\end{equation}

\end{frame}

\subsection{Images with a source}
\begin{frame}
	\frametitle{Including Images}
	\begin{figure}
		\centering
		% \includegraphics[width=0.8\textwidth]{images/tensorflow.pdf}
		\source{Google.com}
		\caption{TensorFlow logo}
	\end{figure}
\end{frame}

\begin{frame}{Basic Blocks Example}
	\begin{block}{Standard block}
		Observation through sound or listening can tell us about our surrounding environment.
	\end{block}
	\begin{alertblock}{Alert block}
		A-weighting mirrors the range of hearing, with frequencies of 20 Hz to 20,000 Hz.
	\end{alertblock}
	\begin{exampleblock}{Example block}
		Recommendations for leisure noise in 2018 were conditional and based on the equivalent sound pressure level during an average 24 hour period in a year without weights for nighttime noise.
	\end{exampleblock}
\end{frame}
\fi

\end{document}
